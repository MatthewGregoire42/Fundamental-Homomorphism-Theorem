\section{Fundamental Homomorphism Theorem}

Now we finally have built all the machinery we need to tackle this theorem. The statement is as follows.

\begin{theorem}[The Fundamental Homomorphism Theorem]
Let $\phi: G \to G'$ be a group homomorphism with kernel $K$. Then $\mu: G/K \to \phi(G)$ defined by $\mu(aK) = \phi(a)$ is an isomorphism. If $\gamma: G \to G/K$ is the canonical homomorphism given by $\gamma(a) = aK$, then $\phi(a) = \mu(\gamma(a))$ for all $a \in G$.
\end{theorem}

\begin{proof}
By Theorem \ref{normalkernel}, $K$ is a normal subgroup of $G$, so it makes sense to discuss $G/K$. First, let's check that $\mu$ is a homomorphism. Let $aK, bK \in G/K$. Because multiplication of cosets of $K$ is well-defined, and $\phi$ is a homomorphism, we can see that
\begin{align*}
    \mu(aKbK) &= \mu(abK) \\
    &= \phi(ab) \\
    &= \phi(a)\phi(b) \\
    &= \mu(aK)\mu(bK)\mathrm{,}
\end{align*}
so the homomorphism property is satisfied. Suppose that $\mu(aK) = \mu(bK).$ Therefore $\phi(a) = \phi(b)$, and because $\phi$ is a homomorphism,
\begin{align*}
    \phi(a)\inv{\phi(b)} &= e' \\
    \phi(a)\phi(\inv b) &= e' \\
    \phi(a\inv b) &= e'\mathrm{.}
\end{align*}
Therefore $a\inv b \in \ker(\phi) = K$. By Theorem \ref{cosets}, this implies that $aK = bK$, so $\mu$ is injective. Now let $\phi(a) \in \phi(G)$. Clearly $\mu(aK) = \phi(a)$, so $\mu$ is also surjective. Therefore $\mu$ is indeed an isomorphism.

To check final property, let $a \in G$. Then $\mu(\gamma(a)) = \mu(aK) = \phi(a)$, so the theorem is proven.

\end{proof}

Essentially this proof tells us the following: if we are given a homomorphism $\phi$ with a kernel $K$, which can be symbolized as the following:

\begin{center}
\begin{tikzcd}
    G \arrow{r}{\phi} & G'
\end{tikzcd}
\end{center}

Then we can always ``factor'' this map as the isomorphism $\mu$ applied to the surjecive map $\gamma$. We can also consider the identity map $\lambda: \phi(G) \to G'$ defined by $\lambda(\phi(a)) = \phi(a)$ as a trivial injective homomorphism between $\phi(G)$ and $G'$. So we can ``factor'' $\phi$ completely as:

\begin{center}
\begin{tikzcd}[column sep=large]
    G \arrow{r}{\gamma}[below]{\mathrm{surjection}} & G/K \arrow{r}{\mu}[below]{\mathrm{bijection}} & \phi(G) \arrow{r}{\lambda}[below]{\mathrm{injection}} & G'
\end{tikzcd}
\end{center}

\begin{example}
Let $\phi$ be the homomorphism from $\z$ to $\zn[4]$ defined in Example \ref{zn_homomorphism}. We know that the kernel of $\phi$ is $4\z$, and also $\phi$ clearly maps $\z$ onto $\zn[4]$. Therefore by the fundamental homomorphism theorem, $\z/4\z$ is isomorphic to $\zn[4]$. This is the elegant algebraic way to \textit{define} $\zn[4]$, as a group of cosets of $4\z$ (which are also congruence classes of $\z$ modulo $4$).
\end{example}

\begin{example}
Define $\phi: D_3 \to \zn[2]$ by:
\begin{align*}
    \phi(e) = \phi(\rho_1) = \phi(\rho_2) &= 0 & \phi(\mu_1) = \phi(\mu_2) = \phi(\mu_3) &= 1.
\end{align*}
It's easy to check that this is indeed a homomorphism by looking at Table \ref{cayleyD3}. The product of two rotations or two reflections is a rotation, and the product of a rotation and a reflection is a reflection. Thus the homomorphism property is satisfied. We can see that $\ker \phi = \{e, \rho_1, \rho_2\}$. This gives us another proof that this is a normal subgroup of $D_3$ by Theorem \ref{normalkernel}. If $K = \ker\phi$, then $D_3/K$ is isomorphic to $\phi(D_3) = \zn[2]$. We compare their Cayley tables in Figure \ref{d3z2} to show that these really are the same group with relabeled elements.

\begin{figure}[ht]
\centering
\begin{tabular}{c | c c}
    $\circ$  & $K$ & $\mu_1K$ \\
    \hline
    $K$      & $K$ & $\mu_1K$ \\
    $\mu_1K$ & $\mu_1K$ & $K$
\end{tabular}
\quad
\begin{tabular}{c | c  c}
    $+$ & $0$ & $1$ \\
    \hline
    $0$ & $0$ & $1$ \\
    $1$ & $1$ & $0$
\end{tabular}
\caption{Cayley tables for $D_3/K$ and $\zn[2]$}
\label{d3z2}
\end{figure}

\end{example}