\pagebreak

\section*{Afterword}
\markright{}
\addcontentsline{toc}{section}{Introduction}

Thank you for taking this journey all the way to the end! Hopefully at this point you have a deeper appreciation for this piece of mathematics. As noted in the introduction, the culmination of this document is sometimes introduced as the first in a set of group isomorphism theorems. The other theorems require a little more machinery, but if you understood everything in this document the other isomorphism theorems shouldn't be too daunting.

This result is an incredibly important step in understanding the theory of groups. Moreover, the fundamental homomorphism theorem presented here is specific to group homomorphisms. In the context of more advanced algebraic structures, such as rings and vector fields, similar results hold. Understanding this theorem is a stepping stone to those more advanced topics.

I'd like to show thanks to John B. Fraleigh for his textbook \textit{A First Course in Abstract Algebra}. I used the seventh edition of this textbook as the source for most of this material. Despite the poor coverage of the fundamental homomorphism theorem, this really is a fantastic book for an introduction to algebra. In addition, Charles C. Pinter's \textit{A Book of Abstract Algebra} was an invaluable resource for me. Its scope isn't as deep as Fraleigh's text, but the explanations in Pinter's book are highly accessible, and the exercises are well-organized. I'd recommend either of these books as a next step in learning about algebraic structures.