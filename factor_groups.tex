\section{Factor Groups}

At this point, all of the concepts we've defined start to come together. First, we need just a little more machinery to operate on cosets.

\begin{theorem}
Let $H$ be a subgroup of $G$, and consider two left cosets $aH$ and $bH$ of $H$. Then the operation $(aH)(bH) = (ab)H$ is well-defined if and only if $H$ is a normal subgroup.
\end{theorem}

Before proving this, let's think about what we need to show. Left coset multiplication, which is a new operation on cosets, is well-defined as a binary operation if and only if it satisfies Definition \ref{binaryoperation}. That is, it needs to map each pair of left cosets to a \textit{unique} corresponding coset. Recall that $aH = cH$ is equivalent to $c \in aH$ by Theorem \ref{cosets}. Therefore, left coset multiplication is well-defined if and only if, for all $ah_1 \in aH$ and $bh_2 \in bH$,
\begin{equation*}
    (aH)(bH) = (ah_1H)(bh_2H) = (ab)H = (ah_1bh_2)H.
\end{equation*}
With this reformulation, we can now proceed to the proof.

\begin{proof}
Assume that the operation is well-defined. We want to show that $H$ is normal. Let $a \in G$. We'll show that $aH = Ha$.

Let $x \in aH$, so $xH = aH$. Computing with our operation, which by assumption is well-defined, we have
\begin{equation*}
    (xH)(\inv a H) = (x\inv a)H.
\end{equation*}
But we could also compute the same product with different representatives, as the following.
\begin{equation*}
    (aH)(\inv a H) = (a \inv a)H = eH = H
\end{equation*}
Therefore $(x\inv a)H = H$. Using the first bulleted property of Theorem \ref{cosets}, this implies that $x\inv a \in H$, so there exists an $h \in H$ such that $x\inv a = h$, or $x = ha \in Ha$. Therefore $aH \subseteq Ha$.

Similarly, let $x \in Ha$. By the same reasoning, we have
\begin{align*}
    (H\inv a)(Hx) &= (H\inv a)(Ha) \\
    H(\inv a x) &= H.
\end{align*}
Therefore $\inv a x = h$ for some $h \in H$, which means that $x = ah \in aH$, so $Ha \subseteq aH$. Therefore $aH = Ha$, and $H$ is normal.

Conversely, suppose that $H$ is normal. We need to show that coset multiplication is well-defined. Let $aH$ and $bH$ be cosets, and let $ah_1 \in aH$ and $ah_2 \in bH$. Then our operation states that $(ah_1)H(bh_2)H = (ah_1bh_2)H$. Because $H$ is normal, we know that $Hb = bH$, so $h_1 b = b h_3$ for some $h_3 \in H$. Therefore, $a\left(h_1b\right) h_2 = ab h_3 h_2$. So we have
\begin{align*}
    \left(ah_1 H\right)\left(bh_2 H\right) &= \left(ah_1bh_2 H\right) \\
    &= ab\left(h_3 h_2\right)H \\
    &= ab H.
\end{align*}
This means coset multiplication doesn't depend on choices of representatives, and is therefore well-defined.

\end{proof}

\begin{example}
Consider the subgroup $H = \{e, \mu_2\}$ of $D_3$. As shown in Example \ref{not_normal_d3}, $H$ is not normal in $D_3$. Therefore multiplication of cosets of $H$ is not well-defined. Indeed, we can see that
\begin{align*}
    \rho_1H = \mu_3H &= \{\rho_1, \mu_3\}, & \mu_1H = \rho_2H &= \{\mu_1, \rho_2\}.
\end{align*}
However, if we attempt to multiply these cosets using the operation defined above, we have
\begin{align*}
    \left(\rho_1H\right)\left(\mu_1H\right) &= \left(\rho_1 \mu_1\right)H \\
    &= \mu_2 H \\
    &= H,
\end{align*}
while choosing different representatives, we have
\begin{align*}
    \left(\mu_3H\right)\left(\rho_2H\right) &= \left(\mu_3 \rho_2\right)H \\
    &= \mu_1 H \\
    &\neq H.
\end{align*}
From this example we can see that choosing different representatives of the cosets actually changes the output! So coset multiplication of $H$ is indeed not well-defined.
\end{example}

You might wonder what the point of defining an operation on the cosets of a subgroup might be. As a matter of fact, if our coset multiplication is well-defined, the cosets themselves form a group! We can often think of the operations on cosets as operations of certain types or classes of elements in the original group.

\begin{theorem}
Let $N$ be a normal subgroup of $G$. Then the set of cosets of $N$ form a group under coset multiplication. We call this group a \textbf{factor group}, or \textbf{quotient group}, denoted $G/N$. This is read as ``$G$ mod $N$'' or ``$G$ over $N$.''
\end{theorem}

\begin{proof}
Since $N$ is a normal subgroup of $G$, coset multiplication is well-defined. Let $aN, bN, cN \in G/N$. $eN = N$ serves as the identity element in $G/N$, because $(eN)(aN) = (ea)N = aN = (ae)N = (aN)(eN)$. In addition, $\left(\inv a N\right)(aN) = \left(\inv a a\right)N = eN = \left(a \inv a\right)N = (aN)\left(\inv a N\right)$, so each coset has its own inverse. Finally, we have
\begin{align*}
    (aN)\left[(bN)(cN)\right] &= aN(bcN) = a(bc)N = (ab)cN \\
    &= (abN)cN = \left[(aN)(bN)\right]cN.
\end{align*}

So associativity in $G/N$ follows from associativity in $G$. Therefore $G/N$ is a group.

\end{proof}

\begin{example}
Consider the subgroup $4\z$ of $\z$ under addition. Because $\z$ is abelian, every subgroup of $\z$ is normal. Therefore $\z/4\z$ is a group whose elements are the cosets of $4\z$, namely $0+4\z$, $1+4\z$, $2+4\z$, and $3+4\z$. The Cayley table for this factor group is shown in Figure \ref{z4z_cayley}. Because $\z$ and its subgroups are additive, we write cosets additively and speak of coset \textit{addition}.

\begin{figure}[ht]
\centering
\begin{tabular}{c | c c c c}
    $+$     & $0+4\z$ & $1+4\z$ & $2+4\z$ & $3+4\z$ \\
    \hline
    $0+4\z$ & $0+4\z$ & $1+4\z$ & $2+4\z$ & $3+4\z$ \\
    $1+4\z$ & $1+4\z$ & $2+4\z$ & $3+4\z$ & $0+4\z$ \\
    $2+4\z$ & $2+4\z$ & $3+4\z$ & $0+4\z$ & $1+4\z$ \\
    $3+4\z$ & $3+4\z$ & $0+4\z$ & $1+4\z$ & $2+4\z$
\end{tabular}
\caption{}
\label{z4z_cayley}
\end{figure}
\end{example}

Finally, let's connect the idea of a factor group to homomorphisms from the previous section.

\begin{theorem}
Let $G$ be a group with a normal subgroup $N$, and let $\phi: G \to G/N$ be defined by $\phi(a) = aN$. Then $\phi$ is a homomorphism, sometimes called the \textit{natural} or \textit{canonical} homomorphism between these groups.
\end{theorem}

\begin{proof}
Let $a, b \in G$. Because $N$ is normal, coset multiplication is well-defined. Therefore $\phi(ab) = abN = (aN)(bN) = \phi(a)\phi(b)$, so the homomorphism property holds.

\end{proof}