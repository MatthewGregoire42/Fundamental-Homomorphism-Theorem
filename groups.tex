\section{Groups}

First, here are the basic preliminaries we need to start talking about groups.

\begin{definition}
\label{binaryoperation}
Let $G$ be a set. A \textbf{binary operation} on $G$ is a function $*: G \times G \to G$.
\end{definition}

If in $G$, we have $*(a, b) = c$, we often write $a * b = c$. In fact, we can often drop the operation symbol and write this as $ab = c$, if no confusion will result. For now, we'll continue explicitly writing the operation for clarity.


\begin{definition}
A \textbf{group} is a set G with a binary operation $*: G \times G \to G$ that satisfies the following properties:
\begin{itemize}
    \item There exists an element $e \in G$ such that, for all $a \in G$, $a*e = e*a = a$. Here $e$ is referred to as the \textbf{identity element} for $*$.
    \item For all $a \in G$, there exists an $\inv{a} \in G$ such that $a*\inv{a} = \inv{a}*a = e$. We call $\inv{a}$ the \textbf{inverse} of $a$.
    \item For all $a, b, c \in G$, $a * (b * c) = (a * b) * c$. In this case, we say $*$ is an \textbf{associative} operation.
\end{itemize}
\end{definition}

We'll give a few examples of groups to get a feel for their structure.

\begin{example}
The integers $\z$ under the usual addition ($+$) form a group. The element $0$ acts as the identity, and for any $a \in \z$, $-a$ acts as its additive inverse. Addition is associative, so this is indeed a group. For similar reasons, the sets $\mathbb{Q}$, $\mathbb{R}$, and $\mathbb{C}$ are also groups under addition.
\end{example}

\begin{example}
The integers $\zn[n]$ of remainders modulo $n$ also form a group under addition. We'll be pedantically formal here for the sake of demonstrating the group axioms. The elements of $\zn[n]$ are defined as the integers $x$ such that $0 \leq x < n$, and addition within $\zn[n]$, for now given the symbol $\tilde +$, is defined as follows: if $a + b = nq + r$ in $\z$ by the division algorithm, then $a \np b = r$. Since $0 \leq r < n$, this is also an element of $\zn[n]$, so this is a well-defined binary operation.

$0$ is again the identity, because $a + 0 = 0 + a = n\cdot 0 + a$, so $a \np 0 = 0 \np a = a$. Also, the equation $a + (n-a) = (n-a) + a = n\cdot 1 + 0$ shows that $n-a$ is the inverse of $a$. Finally, to show associativity, just note that $a + (b + c) = (a + b) + c$ is true for any integers $a$, $b$, and $c$, so therefore adding any elements $a, b, c \in \zn[n]$ must yield the same remainder.
\end{example}

Whew! Most of the time there's no need to be that formal. The groups $\zn[n]$ are rather intuitive, so there's no need to get bogged down in this formalism. The real punchline is that this set of numbers is a group, and operating within $\zn[n]$ should naturally feel like operating on congruence classes of integers modulo $n$.

In order to tackle the fundamental homomorphism theorem, we need to first build up a litte machinery in order to understand groups. The next few theorems build only from the definition of a group.

\begin{theorem}
Let $G$ be a group with a binary operation $*$. Then for all $a, b, c \in G$, $a * b = a * c$ implies that $b = c$, and $b*a = c*a$ also implies $b=c$. These are called the \textbf{left and right cancellation laws}.
\end{theorem}

\begin{proof}
Suppose that $a*b = a*c$. We know there exists $\inv{a} \in G$, so therefore:

\begin{center}
    $\inv{a} * (a*b) = \inv{a} * (a*c)$
\end{center}

By the associative property:

\begin{center}
    $(\inv{a} * a) * b = (\inv{a} * a) * c$
\end{center}

By definition of $\inv{a}$:

\begin{center}
    $e * b = e * c$
\end{center}

And finally, by definition of $e$, we have $b = c$. Similarly, if $b*a = c*a$, then $(b*a)*\inv{a} = (c*a)*\inv{a}$. Therefore $b*(a*\inv{a}) = c*(a*\inv{a})$, and $b*e = c*e$, or $b = c$.

\end{proof}

\begin{theorem}
Let $G$ be a group. There is a unique identity element $e \in G$ such that $e*a = a*e = a$ for all $a \in G$. Similarly, for $a \in G$, there is a unique element $\inv{a} \in G$ such that $a*\inv{a} = \inv{a}*a = e$. Also $\left(\inv{a}\right)^{-1} = a$, that is, the inverse of $\inv a$ is $a$.
\end{theorem}

\begin{proof}
Suppose that $e$ and $e'$ are both elements of $G$ with the given property. Then we have $e = e*e'$, using $e'$ as the identity. But we also have $e*e' = e'$, using $e$ as the identity. Therefore $e = e'$.

Now let $a \in G$. If $a*\inv{a} = \inv{a}*a = e$, and also $a*\inv{\tilde{a}} = \inv{\tilde{a}}*a = e$, then we have:

\begin{center}
    $a*\inv{a} = e = a*{\inv{\tilde{a}}}$
\end{center}

And by cancellation, $\inv{a} = \inv{\tilde{a}}$. For the last property, we only need to note that $\left(\inv{a}\right)^{-1}$ is (by definition) the unique element in $G$ such that $\left(\inv{a}\right)^{-1} * \inv{a} = \inv{a} * \left(\inv{a}\right)^{-1} = e$. By the equation $a * \inv{a} = \inv{a} * a = e$, we see that this unique element is indeed $a$.

\end{proof}

Now, most operations we're familiar with aren't just associative. They have other nice properties as well. The next few definitions solidify this conceptually.

\begin{definition}
Let $*$ be a binary operation on a set $A$. \extra If for all $a, b \in A$, we have $a * b$ = $b * a$, then $*$ is a \textbf{commutative} operation.
\end{definition}

\begin{definition}
Let $G$ be a group with a binary operation $*$. \extra If $*$ is commutative, then $G$ is an \textbf{abelian group}. Otherwise, $G$ is \textbf{non-abelian}.
\end{definition}

\begin{example}
The familiar properties of integer addition show that $\z$ and $\zn[n]$ for any $n$ are abelian groups.
\end{example}

\begin{example}
Let $F$ be the set of all invertible functions from $\mathbb{R}$ to $\mathbb{R}$, and consider the operation of function composition defined on $F$. Here $i(x) = x$ serves as the identity function, and each function in $F$ has an inverse in $F$ by construction. Function composition is associative as well. Finally, if $f, g \in F$, then $\inv{(f \circ g)} = \inv g \circ \inv f$, because: 
\begin{align*}
\left[(f \circ g) \circ \left(\inv g \circ \inv f\right)\right]x &= f\left(g\left(\inv g\left(\inv f \left(x\right)\right)\right)\right) \\
 &= f\left(\left(g \circ \inv g\right)\left(\inv f(x)\right)\right) \\
 &= f\left(i\left(\inv f(x)\right)\right) \\
 &= \left(f \circ \inv f\right)(x) \\
 &= i(x)\mathrm{,}
\end{align*}
which is the identity map. It can be similarly shown that $\inv g \circ \inv f$ is the left inverse of $f \circ g$ as well. Therefore $F$ forms a group under function composition. Take $f(x) = x+1$ and $g(x) = x^3$. Both of these are invertible functions. We also see that $(f \circ g)(x) = x^3 + 1$, while $(g \circ f)(x) = (x+1)^3$. Therefore $F$ is not an abelian group.
\end{example}

For the rest of this section, we'll develop one simple example of a non-abelian group. Consider an equilateral triangle with labeled vertices:

\dihedral{1}{2}{3}

Now we ask ourselves: how we can transform this triangle so that it lands back on itself, potentially with the vertices rearranged? We can do no operation to the triangle, which won't permute the vertices. We can also reflect through an axis going through one vertex, and we can rotate the triangle by $60^\circ$ either clockwise or counterclockwise. This gives six possible operations.