\section{Groups}

First, here are the basic preliminaries we need to start talking about groups.

\begin{definition}
\label{binaryoperation}
Let $G$ be a set. A \textbf{binary operation} on $G$ is a function $*: G \times G \to G$.
\end{definition}

If in $G$, we have $*(a, b) = c$, we usually write $a * b = c$. In fact, we can often drop the operation symbol and write this as $ab = c$, if no confusion will result. For now, we'll continue explicitly writing the operation for clarity.


\begin{definition}
A \textbf{group} is a set G along with a binary operation $*$ that satisfies the following properties:
\begin{itemize}
    \item There exists an element $e \in G$ such that, for all $a \in G$, $a*e = e*a = a$. Here $e$ is referred to as the \textbf{identity element} for $*$.
    \item For all $a \in G$, there exists an $\inv{a} \in G$ such that $a*\inv{a} = \inv{a}*a = e$. We call $\inv{a}$ the \textbf{inverse} of $a$.
    \item For all $a, b, c \in G$, $a * (b * c) = (a * b) * c$. In this case, we say $*$ is an \textbf{associative} operation.
\end{itemize}
\end{definition}

We'll give a few examples of groups to get a feel for their structure.

\begin{example}
The integers $\z$ under the usual addition ($+$) form a group. The element $0$ acts as the identity, and for any $a \in \z$, $-a$ acts as its additive inverse. Addition is associative, so this is indeed a group. For similar reasons, the sets $\mathbb{Q}$, $\mathbb{R}$, and $\mathbb{C}$ are also groups under addition.
\end{example}

\begin{example}
The integers $\zn[n]$ of remainders modulo $n$ also form a group under addition. We'll be pedantically formal here for the sake of demonstrating the group axioms. The elements of $\zn[n]$ are defined as the integers $x$ such that $0 \leq x < n$, and addition within $\zn[n]$, for now given the symbol $\tilde +$, is defined as follows: if $a + b = nq + r$ in $\z$ by the division algorithm, then $a \np b = r$ in $\zn[n]$. Since $0 \leq r < n$, this is also an element of $\zn[n]$, so this is a well-defined binary operation.

$0$ is again the identity, because $a + 0 = 0 + a = n\cdot 0 + a$, so $a \np 0 = 0 \np a = a$. Also, the equation $a + (n-a) = (n-a) + a = n\cdot 1 + 0$ shows that $n-a$ is the inverse of $a$. Finally, to show associativity, just note that $a + (b + c) = (a + b) + c$ is true for any integers $a$, $b$, and $c$, so therefore adding any elements $a, b, c \in \zn[n]$ must yield the same remainder.
\end{example}

Whew! Most of the time there's no need to be that formal. The groups $\zn[n]$ are rather intuitive, so there's no need to get bogged down in this formalism from this point forward. The real punchline is that this set of numbers is a group, and operating within $\zn[n]$ should naturally feel like operating on congruence classes of integers modulo $n$.

In addition to to simply defining groups, we need to build up a litte machinery in order to understand their structure. The next few theorems build only from the definition of a group.

\begin{theorem}
Let $G$ be a group with a binary operation $*$. Then for all $a, b, c \in G$, $a * b = a * c$ implies that $b = c$, and $b*a = c*a$ also implies $b=c$. These are called the \textbf{left and right cancellation laws}, respectively.
\end{theorem}

\begin{proof}
Suppose that $a*b = a*c$. We know there exists $\inv{a} \in G$, so therefore:

\begin{center}
    $\inv{a} * (a*b) = \inv{a} * (a*c)$
\end{center}

By the associative property:

\begin{center}
    $(\inv{a} * a) * b = (\inv{a} * a) * c$
\end{center}

By definition of $\inv{a}$:

\begin{center}
    $e * b = e * c$
\end{center}

And finally, by definition of $e$, we have $b = c$. Similarly, if $b*a = c*a$, then $(b*a)*\inv{a} = (c*a)*\inv{a}$. Therefore $b*(a*\inv{a}) = c*(a*\inv{a})$, and $b*e = c*e$, or $b = c$.

\end{proof}

\begin{theorem}
Let $G$ be a group. There is a unique identity element $e \in G$ such that $e*a = a*e = a$ for all $a \in G$. Similarly, for $a \in G$, there is a unique element $\inv{a} \in G$ such that $a*\inv{a} = \inv{a}*a = e$. Also $\left(\inv{a}\right)^{-1} = a$, that is, the inverse of $\inv a$ is $a$.
\end{theorem}

\begin{proof}
Suppose that $e$ and $e'$ are both elements of $G$ that act as an identity. Then we have $e = e*e'$, using $e'$ as the identity. But we also have $e*e' = e'$, using $e$ as the identity. Therefore $e = e'$.

Now let $a \in G$. If $a*\inv{a} = \inv{a}*a = e$, and also $a*\inv{\tilde{a}} = \inv{\tilde{a}}*a = e$, then we have:

\begin{center}
    $a*\inv{a} = e = a*{\inv{\tilde{a}}}$
\end{center}

And by cancellation, $\inv{a} = \inv{\tilde{a}}$. For the last property, we only need to note that $\left(\inv{a}\right)^{-1}$ is (by definition) the unique element in $G$ such that $\left(\inv{a}\right)^{-1} * \inv{a} = \inv{a} * \left(\inv{a}\right)^{-1} = e$. By the equation $a * \inv{a} = \inv{a} * a = e$, we see that this unique element is indeed $a$.

\end{proof}

Now, most operations we're familiar with are not simply associative. They have other nice properties as well. The next few definitions solidify one of these properties, which should be fairly intuitive.

\begin{definition}
Let $*$ be a binary operation on a set $A$. \extra If for all $a, b \in A$, we have $a * b$ = $b * a$, then $*$ is a \textbf{commutative} operation.
\end{definition}

\begin{definition}
Let $G$ be a group with a binary operation $*$. \extra If $*$ is commutative, then $G$ is an \textbf{abelian group}. Otherwise, $G$ is \textbf{nonabelian}.
\end{definition}

\begin{example}
The familiar properties of addition show that $\z$, $\mathbb{Q}$, $\mathbb{R}$, $\mathbb{C}$, and $\zn[n]$ for any $n$ are abelian groups.
\end{example}

\begin{example}
Let $F$ be the set of all invertible functions from $\mathbb{R}$ to $\mathbb{R}$, and consider the operation of function composition defined on $F$. Here $i(x) = x$ serves as the identity function, and each function in $F$ has an inverse in $F$ by construction. Composition of invertible functions is associative as well. Finally, if $f, g \in F$, then $\inv{(f \circ g)} = \inv g \circ \inv f$, because: 
\begin{align*}
\left[(f \circ g) \circ \left(\inv g \circ \inv f\right)\right]x &= f\left(g\left(\inv g\left(\inv f \left(x\right)\right)\right)\right) \\
 &= f\left(\left(g \circ \inv g\right)\left(\inv f(x)\right)\right) \\
 &= f\left(i\left(\inv f(x)\right)\right) \\
 &= \left(f \circ \inv f\right)(x) \\
 &= i(x)\mathrm{,}
\end{align*}
which is the identity map. It can be similarly shown that $\inv g \circ \inv f$ is the left inverse of $f \circ g$ as well. Therefore $F$ forms a group under function composition. Take $f(x) = x+1$ and $g(x) = x^3$. Both of these are invertible functions. We also see that $(f \circ g)(x) = x^3 + 1$, while $(g \circ f)(x) = (x+1)^3$. Therefore $F$ is not an abelian group.
\end{example}

Now \extra to get more of a feel for the structure of a group, we can introduce \textbf{Cayley tables} (or \textbf{Cayley diagrams}) for finite groups. The Cayley table for $\zn[4]$ is given in Figure \ref{z4}. The rows and columns are titled with elements of the group, and in row $a$ and column $b$ we put the element $a + b$. Note that this is the same as $b + a$ because $\zn[4]$ is abelian, but this is not always the case. For a general group, the entry in row $a$ and column $b$ could be different from the entry in row $b$ and column $a$.

\begin{figure}[ht]
\centering
\begin{tabular}{c | c c c c}
    $+$ & $0$ & $1$ & $2$ & $3$ \\
    \hline
    $0$ & $0$ & $1$ & $2$ & $3$ \\
    $1$ & $1$ & $2$ & $3$ & $0$ \\
    $2$ & $2$ & $3$ & $0$ & $1$ \\
    $3$ & $3$ & $0$ & $1$ & $2$
\end{tabular}
\caption{Cayley table for $\zn[4]$.}
\label{z4}
\end{figure}

For the rest of this section, we'll develop one extended example of a nonabelian group. Consider an equilateral triangle with labeled vertices, shown in Figure \ref{triangle}.
\begin{figure}[ht]
\centering
\dihedral{1}{2}{3}
\caption{}
\label{triangle}
\end{figure}
This might seem a strange thing to consider. But now we ask ourselves: how we can transform this triangle so that it lands back on itself, potentially with the vertices rearranged? We can leave the triangle alone, which won't permute the vertices at all. We can also reflect through an axis going through one vertex, and we can rotate the triangle by $60^\circ$ either clockwise or counterclockwise. This gives six possible transformations, all shown below.

\begin{figure}[ht]
\centering
\begin{tabular}{c c c}
    \dihedral{1}{2}{3} & \dihedral[r1]{3}{1}{2} & \dihedral[r2]{2}{3}{1} \\
    \dihedral[m1]{1}{3}{2} & \dihedral[m2]{3}{2}{1} & \dihedral[m3]{2}{1}{3}
\end{tabular}
\caption{The six symmetries of an equilateral triangle.}
\label{d3}
\end{figure}

From left to right and top to bottom as shown in Figure \ref{d3}, let's give these transformations of the triangle the names $e$, $\rho_1$, $\rho_2$, $\mu_1$, $\mu_2$, and $\mu_3$ respectively, where $\rho$ and $\mu$ are suggesting \textit{rotations} and \textit{mirrorings}, respectively. The crucial step is to \textit{consider these transformations as elements of a group}, where the group operation is composition of transformations. This is a well-defined binary operation, because these six positions exhaust all possible orientations of the vertices. Therefore every two transformations applied in succession will result in another positioning of vertices listed above. Here $e$ acts as the identity element. It's easy to see that the two opposite rotations are inverses of each other, and each reflection is its own inverse. Finally, each transformation can be viewed in a natural way as a bijection between $\{1, 2, 3\}$ and itself: just take the image of $i$ as the label of the vertex that lands in position $i$. The composition of bijections from a set to itself is associative, so composition of transformations is associative. Therefore this set is a group under composition. We'll call this group $D_3$, the \textbf{dihedral group on 3 vertices}.

Last but not least, the Cayley table for $D_3$ is shown in Figure \ref{cayleyD3}. Notice that this group is nonabelian, as promised! This group has many interesting symmetries, and will serve as our prototypical example of a nonabelian group.

\begin{figure}[ht]
\centering
\begin{tabular}{c | c c c c c c c}
    $\circ$  & $e$      & $\rho_1$ & $\rho_2$ & $\mu_1$  & $\mu_2$  & $\mu_3$ \\
    \hline
    $e$      & $e$      & $\rho_1$ & $\rho_2$ & $\mu_1$  & $\mu_2$  & $\mu_3$ \\
    $\rho_1$ & $\rho_1$ & $\rho_2$ & $e$      & $\mu_2$  & $\mu_3$  & $\mu_1$ \\
    $\rho_2$ & $\rho_2$ & $e$      & $\rho_1$ & $\mu_3$  & $\mu_1$  & $\mu_2$ \\
    $\mu_1$  & $\mu_1$  & $\mu_3$  & $\mu_2$  & $e$      & $\rho_2$ & $\rho_1$\\
    $\mu_2$  & $\mu_2$  & $\mu_1$  & $\mu_3$  & $\rho_1$ & $e$      & $\rho_2$\\
    $\mu_3$  & $\mu_3$  & $\mu_2$  & $\mu_1$  & $\rho_2$ & $\rho_1$ & $e$     \\
\end{tabular}
\caption{Cayley table for $D_3$.}
\label{cayleyD3}
\end{figure}
