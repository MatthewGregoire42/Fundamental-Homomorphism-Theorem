\section{Homomorphisms}

Now that we have a good understanding of ways to classify group structures, we'd like to have some way to relate the structure of one group to the structure of another. Homomorphisms are exactly the tool we need. 

\begin{definition}
Let $G$ and $G'$ be groups. A \textbf{homomorphism} between $G$ and $G'$ is a function $\phi: G \to G'$ such that the \textit{homomorphism property}
\begin{equation*}
    \phi(ab) = \phi(a)\phi(b)
\end{equation*}
is satisfied for all $a, b \in G$.
\end{definition}

The important thing to realize here is that the multiplication on the left-hand side is happening within $G$, and the multiplication on the right-hand side is happening within $G'$. If we explicitly write these group operations as $*$ and $\star$, respectively, the homomorphism property above can be written as

\begin{equation*}
    \phi(a*b) = \phi(a) \star \phi(b).
\end{equation*}

\begin{example}
\label{zn_homomorphism}
Consider the groups $\z$ and $\zn[n]$ under addition. We claim that the map $\phi: \z \to \zn[n]$ taking every integer $x$ to its remainder when divided by $n$ is a homomorphism. We only have to check the above property. If $a = q_1n + r_1$ and $b = q_2n + r_2$ using the division algorithm, then
\begin{align*}
    a + b &= q_1n + r_1 + q_2n + r_2 \\
          &= (q_1 + q_2)n + (r_1 + r_2),
\end{align*}
but $r_1 + r_2$ may be greater than $n$. So if $r_1 + r_2 = q_3n + r_3$, then 
\begin{align*}
    a + b &= (q_1 + q_2)n + (q_3n + r_3)  \\
          &= (q_1 + q_2 + q_3)n + r_3.
\end{align*}
Therefore $\phi(a + b) = r_3$. (Note that both of these groups use additive notation.) But we also have $\phi(a) = r_1$ and $\phi(b) = r_2$. Since $\phi(a) + \phi(b)$ is an addition in $\zn[n]$, it's the remainder of $r_1 + r_2$ when divided by $n$, which was defined above to be $r_3$. Therefore $\phi(a) + \phi(b) = r_3 = \phi(a + b)$, so $\phi$ is a homomorphism.
\end{example}

The next theorems will show a few ways in which the structure of $G$ is mapped onto the structure of $G'$ under a homomorphism.

\begin{theorem}
Let $G$ and $G'$ be groups, and let $\phi: G \to G'$ be a homomorphism. Then the following hold:
\begin{itemize}
    \item If $e$ and $e'$ are the respective identities for $G$ and $G'$, then $\phi(e) = e'$.
    \item For all $a \in G$, $\phi(\inv{a}) = \inv{\phi(a)}$.
    \item If $H$ is a subgroup of $G$, then $\phi(H)$ is a subgroup of $G'$.
    \item If $H'$ is a subgroup of $G'$, then $\inv{\phi}(H')$ is a subgroup of $G$.
\end{itemize}
\end{theorem}

\begin{proof}
Let $a \in G$. By our group axioms and application of the homomorphism property, we have:
\begin{equation*}
    e'\phi(a) = \phi(a) = \phi(ea) = \phi(e)\phi(a)
\end{equation*}
Therefore by cancellation, $e' = \phi(e)$.

For the second property, note that $\inv{\phi(a)}$ is the unique element of $G'$ such that $\inv{\phi(a)}\phi(a) = \phi(a)\inv{\phi(a)} = e'$. But we also have the following:
\begin{equation*}
    e' = \phi(e) = \phi(a\inv{a}) = \phi(a)\phi(\inv a)
\end{equation*}
\begin{equation*}
    e' = \phi(e) = \phi(\inv a a) = \phi(\inv a)\phi(a)
\end{equation*}

Therefore we have $\inv{\phi(a)} = \phi(\inv a)$.

Now let $H$ be a subgroup of $G$. First, we know $e \in H$, so therefore $\phi(e) = e' \in \phi(H)$. Let $h', g' \in \phi(H)$. So there exist $h, g \in H$ such that $\phi(h) = h'$ and $\phi(g) = g'$. Because $H$ is a subgroup, $\inv h \in H$, and therefore $\phi(\inv h) = \inv{\phi(h)} = \inv{h'} \in \phi(H)$, so $H'$ is closed under inverses. And because $\phi$ is a homomorphism:
\begin{equation*}
    h'g' = \phi(h)\phi(g) = \phi(hg)
\end{equation*}

But $H$ is a subgroup of $G$, so $hg \in H$, which implies that $h'g' \in \phi(H)$. Therefore $\phi(H)$ is a subgroup of $G'$.

Finally, let $H'$ be a subgroup of $G'$. We need to show that $\inv \phi (H')$ is a subgroup of $G$. Since $H'$ is a subgroup, $e' \in H'$, and because $\phi(e) = e'$, we have $e \in \inv \phi(H')$. Let $a, b \in \inv \phi(H')$. Therefore $\phi(a), \phi(b) \in H'$. Because $H'$ is a subgroup, $\phi(\inv a) = \inv{\phi(a)} \in H'$, meaning that $\inv a \in \inv\phi(H')$, so $\inv\phi(H')$ is closed under inverses. We also know that $\phi(a)\phi(b) = \phi(ab) \in H'$. Therefore $ab \in \inv\phi(H')$, so $\inv\phi(H')$ is closed under multiplication, and is therefore a subgroup of $G$.

\end{proof}

Great! So we can see that homomorphisms (loosely speaking) take identities to identities, inverses to inverses, and subgroups to subgroups. This means we're justified in describing them as maps that preserve structure. There's in fact a stricter kind of structure-preserving map between groups, described in the definition below.

\begin{definition}
Let $G$ and $G'$ be groups. We say a map $\phi: G \to G'$ is a \textbf{group isomorphism} if it is a homomorphism and a bijection.
\end{definition}

If we can find an isomorphism between two groups, then their structures are actually completely identical. The only difference between them is the names of the elements.

\begin{example}
Consider the groups $\zn[3]$ and the subgroup $H = \{e, \rho_1, \rho_2\}$ of $D_3$. We claim that the map $\phi: \zn[3] \to H$ given by
\begin{align*}
    \phi(0) &= e & \phi(1) &= \rho_1 & \phi(2) &= \rho_2
\end{align*}
is an isomorphism. It's clearly a bijection, and we only have to check that it's a homomorphism. There are simpler ways to check this, but the group is small enough that we can simply test every possible input to $\phi$. It helps that we know $\zn[3]$ is abelian, so we can reduce the number of inputs to check by half. We can also check by looking at the Cayley table in Figure \ref{cayleyD3} that $H$ is also abelian.

\begin{center}
    \begin{tabular}{c c c c c c c}
        $a$ & $b$ & $a+b$ & $\phi(a)$ & $\phi(b)$ & $\phi(a)\phi(b)$ & $\phi(a+b)$ \\
        \hline
        $0$ & $0$ & $0$   & $e$       & $e$       & $e$              & $e$        \\
        $0$ & $1$ & $1$   & $e$       & $\rho_1$  & $\rho_1$         & $\rho_1$   \\
        $0$ & $2$ & $2$   & $e$       & $\rho_2$  & $\rho_2$         & $\rho_2$   \\
        $1$ & $1$ & $2$   & $\rho_1$  & $\rho_1$  & $\rho_2$         & $\rho_2$   \\
        $1$ & $2$ & $0$   & $\rho_1$  & $\rho_2$  & $e$              & $e$        \\
        $2$ & $2$ & $1$   & $\rho_2$  & $\rho_2$  & $\rho_1$         & $\rho_1$   \\
    \end{tabular}
\end{center}

The last two columns match, so this is indeed an isomorphism.
\end{example}

For now, note that an isomorphism must be injective, but this need not be the case for homomorphisms. In particular, more than one element of $G$ may be mapped to the identity of $G'$. This concept is important enough that it warrants a definition.

\begin{definition}
Let $\phi: G \to G'$ be a group homomorphism. The set
\begin{equation*}
    \{g \in G \,|\, \phi(g) = e'\}
\end{equation*}
is called the \textbf{kernel} of $\phi$, denoted $\ker \phi$.
\end{definition}

\begin{theorem}
\label{normalkernel}
If $\phi: G \to G'$ is a group homomorphism, then $\ker\phi$ is a normal subgroup of $G$.
\end{theorem}

\begin{proof}
Clearly $e \in \ker\phi$, because $\phi(e) = e'$. If $a, b \in \ker\phi$, then
\begin{center}
    $\phi(\inv a) = \inv{\phi(a)} = \inv{e'} = e'$,
\end{center}
so $\inv a$ is also in $\ker\phi$. Also, by the homomorphism property,
\begin{center}
    $\phi(ab) = \phi(a)\phi(b) = e'e' = e'$,
\end{center}
so $\ker\phi$ is a subgroup of $G$. To show that it's normal, let $g \in G$ and $k \in \ker\phi$. Then we have
\begin{center}
    $\phi(gk\inv g) = \phi(g)\phi(k)\phi(\inv g) = \phi(g) e' \inv{\phi(g)} = e'$,
\end{center}
So $gk\inv g$ is in $\ker\phi$ as well.

\end{proof}

\begin{example}
Let $\phi: \z \to \zn[4]$ be the homomorphism defined in Example \ref{zn_homomorphism}. The kernel of $\phi$ is the set of all $n \in \z$ such that the remainder of $n$ divided by $4$ is $0$. In other words, this is all multiples of four. This is indeed the subgroup $4\z$ of $\z$, and it is automatically normal because $\z$ is abelian.
\end{example}

We'll end this section with a useful way to test if a homomorphism is injective, although we won't be using it after this.

\begin{theorem}
Let \extra $\phi: G \to G'$ be a group homomorphism. Then $\phi$ is injective if and only if $\ker \phi = \{e\}$.
\end{theorem}

\begin{proof}
Suppose that $\phi$ is injective. Then if $\phi(a) = e'$, the fact than $\phi(e) = e'$ proves that $a = e$. Therefore $\ker\phi = \{e\}$. Conversely, if we know $\ker\phi = \{e\}$, suppose that $\phi(a) = \phi(b)$. Then we have
\begin{align*}
    \phi(a)\inv{\phi(b)} &= e' \\
    \phi(a)\phi(\inv b) &= e' \\
    \phi(a \inv b) &= e' \\
\end{align*}
Therefore $a\inv b \in \ker\phi$, so $a\inv b = e$. Multiplying on the right by $b$, we see that $a = b$, so $\phi$ is injective.
\end{proof}

We \extra know that $\phi(G)$ is a subgroup of $G'$ for any homomorphism $\phi: G \to G'$. Since $\phi$ definitely maps $G$ \textit{onto} $\phi(G)$, this means that $G$ is actually isomorphic to $\phi(G)$ if and only if $\phi$ is injective, that is, if and only if $\ker\phi = \{e\}$. While not strictly useful for our purposes, this shows some of the power of kernels of homomorphisms. This theme will be expanded upon through the rest of this document.