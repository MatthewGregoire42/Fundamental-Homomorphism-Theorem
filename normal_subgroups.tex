\section{Normal Subgroups}

Normal subgroups may be the first concept introduced here that seems a little unmotivated. As it turns out, normal subgroups are incredibly relevant to homomorphisms. If you feel like you're missing the forest for the trees, understanding the \textit{statement} of a theorem is more important than understanding its proof, because we'll be building on these concepts from here on out. Feel free to skip a confusing proof and come back to it later.

\begin{definition}
Let $H$ be a subgroup of $G$. We call $H$ a \textbf{normal} subgroup of $G$ if
\begin{center}
    $gH\inv{g} = \{gh\inv{g} \,|\, h \in H\} = H$
\end{center}
for all $g \in G$.
\end{definition}

As it turns out, it doesn't matter much what we take as the definition of a normal subgroup, as the next theorem shows. Textbooks often take any one of the following equivalent characterizations of normal subgroups as the definition.

\begin{theorem}
\label{normaldefs}
Let $H$ be a subgroup of $G$. The following are equivalent:
\begin{itemize}
\item For all $g \in G$, $gH\inv{g} = H$.
\item For all $g \in G$ and $h \in H$, $gh\inv{g} \in H$.
\item For all $g \in G$, $gH = Hg$.
\end{itemize}
\end{theorem}

\begin{proof}
First, assume that $gH\inv{g} = H$. then clearly if $gh_1\inv{g} \in gH\inv{g}$, then there exists an $h_2 \in H$ such that $gh_1\inv{g} = h_2$, so the first implication holds.

Now, assume that the second listed property holds. Let $gh \in gH$. We know that there exists an $h' \in H$ such that $gh\inv{g} = h'$. Therefore $gh = h'g \in Hg$, so $gH \subseteq Hg$. Similarly, let $hg \in Hg$. Applying the assumption with the element $\inv{g} \in G$, we know that there exists an $h' \in H$ such that $\inv{g}h\left(\inv{g}\right)^{-1} = \inv{g}hg = h'$. Therefore $hg = h'g \in Hg$, so $Hg \subseteq gH$. This means that $gH = Hg$.

Finally, let $g \in G$, and assume that $gH = Hg$. Therefore:
\begin{align*}
    \{gh \,|\, h \in H\} &= \{hg \,|\, h \in H\} \\
    \{ghg^{-1} \,|\, h \in H\} &= \{hgg^{-1} \,|\, h \in H\} \\
    \{ghg^{-1} \,|\, h \in H\} &= \{h \,|\, h \in H\} \\
    gHg^{-1} &= H
\end{align*}

So our theorem is proven. We can feel free to use these alternate definitions of normal subgroups interchangeably.

\end{proof}

\begin{example}
Any subgroup of an abelian group is normal. If $H$ is a subgroup of an abelian group $G$, then let $h \in H$ and $g \in G$. We see that $gh\inv g = g \inv g h = eh = h \in H$, so by the second property of Theorem \ref{normaldefs}, $H$ is normal. In particular, this shows that all subgroups of $\z$ and $\zn[n]$ for any $n \in \z^+$ are normal.
\end{example}

\begin{example}
\label{not_normal_d3}
Let $H$ be the subgroup $\{e, \mu_2\}$ of $D_3$, described in Example \ref{d3_subgroups}. Note that
\begin{equation*}
    \rho_1 H = \{\rho_1, \mu_3\}\mathrm{, but}
\end{equation*}
\begin{equation*}
    H \rho_1 = \{\rho_1, \mu_1\}.
\end{equation*}
Therefore $\rho_1 H \neq H\rho_1$, so $H$ is not a normal subgroup of $D_3$. Alternatively, we could note that $\rho_1\mu_2\inv{\rho_1} = \left(\rho_1 \mu_2\right)\rho_2 = \mu_3 \rho_2 = \mu_1 \notin H$ to show that $H$ is not normal in $D_3$.
\end{example}

\begin{example}
\label{normal_d3}
Now let $N$ be the subgroup $\{e, \rho_1, \rho_2\}$ of $D_3$. The left and right cosets are given below.
\begin{align*}
    eN      &= \{e, \rho_1, \rho_2\}   & He &= \{e, \rho_1, \rho_2\} \\
    \mu_1 H &= \{\mu_1, \mu_3, \mu_2\} & H \mu_1 &= \{\mu_1, \mu_2, \mu_3\} \\
\end{align*}
These are the only cosets of $N$ in $D_3$ by Theorem \ref{cosets}. (An identical result holds for right cosets as well.) Therefore we can see that $aN = Na$ for all $a \in N$, so $N$ is normal in $D_3$. This is a special case of the theorem below.
\end{example}

\begin{theorem}
Let \extra $G$ be a group with a subgroup $H$. If $|G|/|H| = 2$, then $H$ is a normal subgroup of $G$.
\end{theorem}

\begin{proof}
We see by the proof of Lagrange's Theorem that there are exactly two cosets of $H$ in $G$. Therefore these are precisely $H$ and $G - H$. If $a \in H$, then clearly $aH$ and $Ha$ are simply $H$ itself. And if $a \notin H$, then $a$ must be in $G-H = aH = Ha$, so $H$ is normal in $G$. 
\end{proof}