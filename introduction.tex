\section*{Introduction}
\markright{}
\addcontentsline{toc}{section}{Introduction}

The goal of this document is to introduce the reader to the fundamental homomorphism theorem for groups, starting from basic definitions. This result is sometimes otherwise known as the first isomorphism theorem. I chose to write this document because of the difficulty I had understanding this theorem in my undegraduate algebra class. It's quite an elegant piece of mathematics, but looking back I was shocked at how poorly it was explained in my textbook.

This document is designed for a reader who has the sole objective of understanding the reasoning behind this theorem, and therefore its scope is narrow. To achieve our goal, we proceed at a breakneck pace most of the way through. If this is your first exposure to this material, make sure to take time to digest each result.

That being said, we would be missing many fundamental (and beautiful) facts about the basics of  group theory if we focus only on the necessary and sufficient definitions and theorems. Therefore we'll motivate these concepts with several examples, and also present some results that aren't strictly required. \extra The examples are clearly labeled if you don't care to read them, and unnecessary results are marked with a spade in the left margin, as shown in this sentence. The hope in marking the unnecessary parts is that, if this is used as a reference material, the reader will know exactly which results are necessary and sufficient to prove the theorem. None of the main content depends on this extra material, but the examples and marked theorems do build on each other from the beginning.