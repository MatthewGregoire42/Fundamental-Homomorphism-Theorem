\documentclass[12pt]{article}
\usepackage[utf8]{inputenc}

\title{The Fundamental Homomorphism Theorem}
\author{Matthew Gregoire}
\date{May 2020}

\usepackage{natbib}
\usepackage{graphicx}
\usepackage{amsthm}

\newcommand{\inv}[1]{#1^{-1}}


\theoremstyle{definition}
\newtheorem{definition}{Definition}[section]
\newtheorem{theorem}{Theorem}[section]

\newenvironment{thm}
{\theoremstyle{definition}
\begin{theorem}}
{\end{theorem}}

\newenvironment{defn}[1][]
{\theoremstyle{definition}
\begin{definition}{}}
{\end{definition}}

\begin{document}

\maketitle

\tableofcontents

\pagebreak

\section{Introduction}

While taking an undergraduate abstract algebra class, the most confusing topic for me was the fundamental homomorphism theorem for groups, otherwise known as the first isomorphism theorem. This result is quite elegant, but looking back I'm shocked at how poorly it was explained in my textbook. This is meant to be a document to build straight up to this theorem from basic definitions.

\section{Groups}

First, here are the basic preliminaries we need to start talking about groups.

\begin{defn}
Let $G$ be a set. A \textbf{binary operation} on $G$ is a function $*: G \times G \to G$.
\end{defn}

If in $G$, we have $*(a, b) = c$, we often write $a * b = c$. In fact, we can often drop the operation symbol and write this as $ab = c$, if no confusion will result. For now, we'll continue explicitly writing the operation for clarity.


\begin{defn}
A \textbf{group} is a set G with a binary operation $*: G \times G \to G$ that satisfies the following properties:
\begin{itemize}
    \item There exists an element $e \in G$ such that, for all $a \in G$, $a*e = e*a = a$. Here $e$ is referred to as the \textbf{identity element} for $*$.
    \item For all $a \in G$, there exists an $\inv{a} \in G$ such that $a*\inv{a} = \inv{a}*a = e$. We call $\inv{a}$ the \textbf{inverse} of $a$.
    \item For all $a, b, c \in G$, $a * (b * c) = (a * b) * c$. In this case, we say $*$ is an \textbf{associative} operation.
\end{itemize}
\end{defn}

In order to tackle the fundamental homomorphism theorem, we need to first build up a litte machinery in order to understand groups. The next few theorems build only from the definition of a group.

\begin{thm}
Let $G$ be a group with a binary operation $*$. Then for all $a, b, c \in G$, $a * b = a * c$ implies that $b = c$, and $b*a = c*a$ also implies $b=c$. These are called the \textbf{left and right cancellation laws}.
\end{thm}

\begin{proof}
Suppose that $a*b = a*c$. We know there exists $\inv{a} \in G$, so therefore:

\begin{center}
    $\inv{a} * (a*b) = \inv{a} * (a*c)$
\end{center}

By the associative property:

\begin{center}
    $(\inv{a} * a) * b = (\inv{a} * a) * c$
\end{center}

By definition of $\inv{a}$:

\begin{center}
    $e * b = e * c$
\end{center}

And finally, by definition of $e$, we have $b = c$. Similarly, if $b*a = c*a$, then $(b*a)*\inv{a} = (c*a)*\inv{a}$. Therefore $b*(a*\inv{a}) = c*(a*\inv{a})$, and $b*e = c*e$, or $b = c$.

\end{proof}

\begin{thm}
Let $G$ be a group. There is a unique element $e \in G$ such that $e*a = a*e = a$ for all $a \in G$. Similarly, for $a \in G$, there is a unique element $\inv{a} \in G$ such that $a*\inv{a} = \inv{a}*a = e$.
\end{thm}

\begin{proof}
Suppose that $e$ and $e'$ are both elements of $G$ with the given property. Then we have $e = e*e'$, using $e'$ as the identity. But we also have $e*e' = e'$, using $e$ as the identity. Therefore $e = e'$.

Now let $a \in G$. If $a*\inv{a} = \inv{a}*a = e$, and also $a*\inv{\tilde{a}} = \inv{\tilde{a}}*a = e$, then we have:

\begin{center}
    $a*\inv{a} = e = a*{\inv{\tilde{a}}}$
\end{center}

And by cancellation, $\inv{a} = \inv{\tilde{a}}$.

\end{proof}

\section{Subgroups and Cosets}

From here on out, we'll use multiplicative notation exclusively when discusing group operations. Also, constantly specifying the name of a group's operation can get tedious, so by abuse of notation, we can feel free to refer to a group $G$ by itself. When we do this, it's implied that $G$ has an associated operation that satisfies the group axioms. Since it's still convenient to give this operation a name, we might as well call it multiplication in most cases. For our purposes, this shouldn't cause any confusion.

Subgroups are still a somewhat intuitive topic. The formal definition is below.

\begin{defn}
Let $G$ be a group, and let $H \subseteq G$. If under the binary operation of $G$, $H$ forms a group, we call $H$ a \textbf{subgroup} of $G$.
\end{defn}

An important and related concept is closure under a specified binary operation.

\begin{defn}
Let $*$ be a binary operation of a set $G$, and let $H \subseteq G$. If for all $a, b \in H$, $a*b$ is also in $H$, then $H$ is \textbf{closed under}, or \textbf{has closure under}, the operation $*$.
\end{defn}

In order for $H$ to be a subgroup of $G$, $H$ needs to satisfy the group axioms. This makes the next theorem feel somewhat tautological, but it provides a methodical way to check if a given subset $H$ is indeed a subgroup of $G$.

\begin{thm}
Let $G$ be a group, and let $H \subseteq G$. Then $H$ is a subgroup of $G$ if and only if:
\begin{itemize}
    \item $e \in H$
    \item For all $a \in H$, $\inv{a} \in H$. (\textbf{Closure under inverses})
    \item For all $a, b \in H$, $ab \in H$. (\textbf{Closure under multiplication})
\end{itemize}
\end{thm}

\begin{proof}
Suppose $H$ is a subgroup of $G$. Then the group axioms hold within $H$, so clearly the first two properties hold. And if $H$ is a group under the operation of $G$, then this operation is a map from $H \times H$ to $H$. The third property follows from this.

Conversely, suppose that the above properties hold for a subset $H$ of $G$. Since we are guaranteed an identity element, and an inverse for every element, we only need to check the associative property. Let $a, b, c \in H$. Then applying the group axioms in $G$, we have $a(bc) = (ab)c$. But this can also be viewed as an equation in $H$, so multiplication is associative within $H$.

\end{proof}

Subgroups also naturally give rise to a discussion of cosets.

\begin{defn}
Let $H$ be a subgroup of a group $G$, and let $a \in G$. The set $\{ah \,|\, h \in H\}$, denoted $aH$, is a subset of $G$. We call this the \textbf{left coset} of $H$ containing $a$. Similarly, the set $Ha = \{ha \,|\, h \in H\}$ is called the \textbf{right coset} of $H$ containing $a$.
\end{defn}

The following theorems and proofs involving left cosets have symmetrical theorems and proofs involving right cosets, so for this document we'll mostly deal with left cosets. Cosets have a number of interesting properties, but proofs about properties of cosets usually aren't illuminating by themselves. Therefore, it's important to have a good intuition for some of their basic properties. We'll give a few of these properties in the following theorems.

\begin{thm}
Let $H$ be a subgroup of $G$, and let $a, b \in G$. The following are equivalent:
\begin{itemize}
\item $a \in bH$
\item $aH = bH$
\item $\inv{b}a \in H$
\end{itemize}
\end{thm}

\begin{proof}

We'll prove these implications in order. First, suppose $a \in bH$. So $a = bh_1$ for some $h \in H$. Let $ah \in aH$. Then $ah = (bh_1)h = b(h_1 h)$. Because $H$ is a subgroup, $h_1 h \in H$, so $b(h_1 h) \in bH$, meaning that $ah \in bH$, or $aH \subseteq bH$. Now let $bh \in bH$. Note that $b = a \inv{h_1}$. Therefore $bh = (a \inv{h_1})h = a (\inv{h_1}h)$. Again, because $H$ is a subgroup, $\inv{h_1} h \in H$, so $bh \in aH$, and $bH \subseteq aH$. Therefore $aH = bH$.

Now suppose that $aH = bH$. If $ah_1 \in aH$, then there exists an $h_2 \in H$ such that $ah_1 = bh_2$. Multiplying on the left by $\inv{b}$ and on the right by $\inv{h_1}$, we see that $\inv{b}a = h_2 \inv{h_1}$. Because $H$ is a subgroup, we therefore have $\inv{b}a \in H$.

Finally, suppose that $\inv{b}a \in H$. So there exists an $h \in H$ such that $\inv{b}a = h$. Multiplying on the left by $b$, this implies that $a = bh \in bH$, proving the theorem.

\end{proof}

Again, the proof of the above theorem isn't terribly important. The important thing to realize is that, in a loose sense, the equivalent statements above can be ``multiplied by $b$'' to get one of the other forms.

\begin{thm}
Let $H$ be a subgroup of a group $G$. Then the set of all left cosets of $H$ partition $G$.
\end{thm}

\begin{proof}
Since $H$ is a subgroup of $G$, we must have $e \in H$, so $H$ is nonempty. Therefore no left coset of $H$ is empty by construction. Let $a \in G$. Since $e \in H$, we have $ae = a \in aH$, so each element is in at least one coset. Suppose $a \in bH$ and $a \in cH$ for $b, c \in G$. By the above theorem, we therefore have $aH = bH$ and $aH = cH$. Set equality is an equivalence relation, so therefore $bH = cH$. This shows that every element of $G$ is in exactly one left coset of $H$.

\end{proof}

% Maybe add something about Theorem of Lagrange here.

\end{document}
